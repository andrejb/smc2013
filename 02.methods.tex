%-----------------------------------------------------------------------------
% __  __      _   _               _     
%|  \/  | ___| |_| |__   ___   __| |___ 
%| |\/| |/ _ \ __| '_ \ / _ \ / _` / __|
%| |  | |  __/ |_| | | | (_) | (_| \__ \
%|_|  |_|\___|\__|_| |_|\___/ \__,_|___/
%-----------------------------------------------------------------------------

\section{Methods}
\label{sec:methods}

In order to meet the needs for real time audio processing, the microcontroller
has to be tweaked so we can capture, process and output analog audio. Each of
these tasks can be performed in a variety of ways, and for this examination we
chose to go with the basic functionalities of the platform.

In this investigation, we used an Arduino Duemilanove with an ATmega328P
microcontroller from Atmel, a very modest version of the platform. It has an
8-bit RISC central processor, operates with a base frequency of 16~MHz, and
has memory capacity of 32~KB for program storage and 2~KB for random access.
From now on, whenever we refer to \emph{the microcontroller}, we are in fact
talking about this specific model from this specific manufacturer.

\subsection{Microcontroller's elements}

\setlength{\itemsep}{0em}

To be able to know how to configure the platform to suit our needs, a general
understanding of the inner workings of a microcontroller is needed. The Atmel
megaAVR series microcontroller is comprised of several components, some
of which are fundamental for our investigation and so will be briefly covered
in this section.

\subsubsection{Clocks}

Many \emph{clocks} provide the frequencies in which the different parts of the
microcontroller work. They are basically either emitters or dividers of square
wave signals that provide the frequency of operation of the CPU, the ADC, the
memory access and other components of the microcontroller. Possible sources of
clock frequencies are crystal and RC oscillators.

A useful concept associated with clocks is the one of a \emph{prescaler}.
Prescalers are dividers for clock frequencies that either actually lower the
frequency of a clock or at least trigger specific interrupts on a (power of
two) fraction of a clock's frequency.

The \emph{system clock} provides the system's base frequency of operation.
Other important clocks are the \emph{I/O clock} and the \emph{ADC clock} used
for feeding a frequency to most of the input/output mechanisms. It is possible
to choose which clock will feed a frequency to some parts of the system, as
well as select prescaler values independently. In our study, we make use of
the \emph{timer clock} prescaler to control the PWM frequency that drives our
DSP mechanism, as we will see in Section \ref{sec:pwm}. 

\subsubsection{Registers and interrupts}

The microcontroller's CPU is comprised of an arithmetic logic unit that works
with 32 \emph{registers} -- portions of memory that provide data for
computation as well as determine the execution flow of the program. An
\emph{interrupt} is an attempt of deviation from the current execution flow
that can be triggered by a variety of events in the system, usually by setting
reference values on specific registers.

In our case, interrupts are of extreme value as they are the low level
structures that allow us to execute code with a somewhat fixed frequency (at
least if we assume that the clock frequencies are indeed constant in relation
with real time).

\subsubsection{Timers/counters}

A \emph{timer}, or \emph{counter}, is a register whose value is automatically
incremented according to a specific clock.
% Esta frase não parece imprescindível...
% A certain timer has a fixed length
% in bits and may have lots of interrupts associated with its behaviour.
When a
counter hits its maximum value it is reset to zero and signals
an overflow interrupt, which may cause a certain function to be called.

Timers are important in the context of DSP because they provide a natural way to
perform many of the DSP chain tasks, as for example to periodically launch the
input signal sampling function (that fills the input buffer) and to emmit a
PWM square wave which, after
% some (analog)
% prefiro ser mais específico, me corrija se estiver errado:
analog low-pass 
filtering (through an integrator),
corresponds to a smooth analog signal.
% causes the digital
% signal to be converted back to analog.
The ATmega328P has two 8 bit counters
and one 16 bit counter, each having different sets of features but all being
capable of doing PWM.

\subsubsection{Input and output pins}

Microcontrollers can receive and emit digital signal through \emph{I/O pins},
which in the case of the Arduino board are conveniently mounted in such a way
that it is easy to plug other components and boards. These pins are read from
and written to according to frequencies governed by different clocks (I/O, ADC
and others).

In principle, the microcontroller pins are designed to work with binary
signals represented by two different voltages (0~V and 5~V with a threshold
value to account for small deviations). Despite
that, I/O pins come equipped with handy mechanisms for sampling band limited
input signals whose voltages vary between the reference extremes, and also for
generating waveforms that, after being filtered, output varying signals of the
same nature. These mechanisms are, respectively, the analog-to-digital
converter (ADC) and the pulse-width modulation (PWM), which will be seen in
the next sections.


\subsubsection{Memory}

The microcontroler has 3 manageable memory spaces for storing the program and
working data, and the following table summarizes the different characteristics
and purposes for each type of memory:

\begin{center}
\begin{tabular}{crcrr}
\toprule
\toprule
\multicolumn{1}{c}{\footnotesize{Type}} &
\multicolumn{1}{c}{\footnotesize{\parbox{2em}{Size (KB)}}} &
\footnotesize{\parbox{4em}{Data persistency}} &
\footnotesize{\parbox{5em}{Write time (clock ticks)}} &
\multicolumn{1}{c}{\footnotesize{\parbox{5em}{Endurance (write/erase cycles)}}} \\
\midrule
Flash  & 32 & yes & 1  & 10,000 \\
SRAM   & 2  & no  & 2  & 10,000 \\
EEPROM & 1  & yes & 30 & 100,000 \\ 
\bottomrule
\end{tabular}
\end{center}

Usually, the Flash memory stores the program, the SRAM memory stores volatile
data used along the computation, and the EEPROM is used for longer-term
storage between working sessions. Notice that the amount of SRAM memory
represents a hard limit for many DSP algorithms. A 512 point lookup table
filled with precalculated sinewave bytes, for example, represents $25\%$
of all available working space. Thus, it might be interesting to store
hardcoded data in the program memory whenever possible if memory working space
is lacking.


\subsection{Audio in: ADC}

Data can flow into the microcontroller in a variety of ways, the most basic
being embedded mechanisms for digital serial communication and analog-to-digital conversion using the input pins. The
former mechanism can feed digital data directly to memory, while the latter
can either read 1 bit from an input pin (as explained in the last section) or sample an
analog value between the reference voltages using 8 or 10 bits resolution.

Rather than providing the microcontroller with digital data, our setup uses
the embedded analog-to-digital conversion to sample an audio signal using
the microcontroller pins' ADC mechanism. This choice was made so the signal
can be directly connected to the microcontroller (i.e. no external device has
to be used for sampling) and we can study the device's performance taking into
account this crucial step in the digital audio processing chain.

The ADC uses a \emph{Sample and Hold} circuit that holds the input voltage at
a constant level until the end of the conversion. This fixed voltage is then
successively compared with reference voltages to obtain a 10 bit
approximation. If a faster conversion is desired, precision can be sacrificed
and the first 8 bits can be read before the last 2 are computed. Conversion
time takes between 13 and 250~$\mu$s, depending on several configuration
parameters that influence the precision of the
result.

As noted before, the ADC mechanism has a dedicated clock to ensure conversion
can occur independently of other microcontroller parts. Also, the mechanism
can be triggered manually (on demand) or automatically (a new conversion
starts as soon as the last one finishes).

\subsection{Audio out: PWM}
\label{sec:pwm}

Once the input signal has been sampled and processed, one way to convert it
back to analog is to use the embedded \emph{pulse width modulation} (PWM) mechanism that is available in some
of the output pins of the microcontroller, followed by an analog filtering
stage. A PWM wave encodes a determined value in the width of a square pulse.
In order to do this, it defines a \emph{duty cycle} as the percentage of time
that the square wave has its maximum value in relation to the total time between square pulses (see Figure \ref{fig:pwmmeu}). The encoding of a value $x$
ranging from $X_1$ to $X_2$ is just the enforcement of a duty cycle with a
percentage equal to $\frac{x}{X_2-X_1}$.

\figura{pwmmeu}{0.40}{Examples of PWM waves with different duty cycles. The
left alignment of the waves expose the use of Fast PWM mode.}

The final analog filtering stage is needed to remove high frequency components
present in the square wave spectrum to reconstruct a band limited signal. In our case,
this filtering is made from a simple RC integrator circuit that stands between the output
pin and a normal speaker.
% TODO: inserir referencia.

The PWM mechanism can operate in different modes which vary according to how
the reference value to be encoded relates with a counter's signal to generate
the output values of the modulated wave. In \emph{Fast PWM} mode, the output
signal is 1 in the beginning of the cycle and is 0 whenever the
reference value becomes smaller than the counter value (see Figure
\ref{fig:pwm2-new}). This mode has the disadvantage of outputing the square pulses
aligned to the left of the PWM cycle, and so the \emph{Phase correct} mode is
available to solve this problem at the expense of cutting the signal
generation frequency in half. It works by making the counter count back to
zero instead of being reset when it hits its maximum value.

\figura{pwm2-new}{0.50}{Time evolution of register values in the PWM mechanism. \texttt{TCNTn} is the value of the counter and \texttt{OCnx} is the value of the output pin. Note how changes in the reference value determine the duty cycle on each wave period.}

The output frequency of the PWM signal is a function of the clock selected to be
used as input for the counter, the counter prescaler value, the size of the
counter (in bits) and the PWM mode. For a $b$ bits counter with input clock of
$f_{clock}$~Hz and prescaler value of $p$, an output pin configured to operate
in fast PWM mode overflows with a frequency of $\frac{f_{clock}}{p \times
2^b}$~Hz. This provides us with a way to output the processed signal while
using the same infrastructure to schedule periodic actions, such as querying
for ADC values and signaling that blocks of samples are ready to be processed.

Also notice that the counter size determines the output signal resolution, as
the duty cycles of the square waves correspond to the ratio between the
current counter value and its maximum possible value. We will see more about
parameters choice for PWM on section \ref{sec:proc-pwm}.

\subsection{Real time processing}

The main constraint in real time DSP is, of course, the amount of time
available for the computation of output samples: they must be ready to be
consumed by the playback hardware or else glitches and other unwanted
artifacts will possibly be introduced in the signal. One round of sample
analysis, processing and calculation of a new sample is called a \emph{DSP sample
cycle}. Many algorithms, though, operate in blocks of samples, consuming and
producing a whole block of samples in each round.
% A frase abaixo apenas "soletra" a fórmula que aparece em seguida (desnecessário)
% Naturally, in these cases
% the period of the DSP cycle is the period of one sample times the size of the
% DSP block.
If the DSP block has $N$ samples and the sample rate is
$R$~Hz, then the DSP cycle is given by $T_{DSP}=\frac{N}{R}$ seconds.

In order to implement this behaviour in the microcontroller, we have to find a
way to (1) accumulate input samples in a buffer, (2) schedule a periodic call
to a function that will process the samples in this buffer, and (3) output
modified samples in a timely fashion. Components are at hand: ADC for reading
the input signal, counters and their interrupts for running periodic tasks, and PWM for outputing
the resulting signal. In addition, the Arduino library provides a \texttt{loop()}
function that is called repeatedly which we can use to process the block of
samples when it becomes available.

As we saw in section \ref{sec:pwm}, the
PWM mechanism provides an overflow interrupt frequency that may be used to
schedule a function for periodic execution. In our setup we use this mechanism
to periodically read samples from the ADC mechanism and accumulate them in an input
buffer, while also writing the computed samples from the last DSP cycle to the PWM output buffer. In
this same function, whenever the buffer is full and ready to be processed,
a flag is set and the \texttt{loop()} function is released to work
on the samples.


\subsection{Implementation}

Putting all the elements together is a matter of choosing the right parameters
for configuring different parts of the microcontroller.

\subsubsection{ADC parameters}

ADC conversion takes about 14.5 ADC clocks, including sample-and-hold time. If the CPU clock
frequency is 16~MHz and the ADC prescaler has a value of $p$, then the ADC clock
period is $p/16$ and the conversion period is then $T_\text{conv} = 14.5 \times p / 16$.
Below we can see a table with the theoretical values for the conversion period $T_\text{conv}$ for
all prescaler values available and also the results $\tilde{T}_\text{conv}$ of measured conversion
times using each prescaler value. Also depicted
in the table are the measured conversion frequencies $\tilde{f}_\text{conv}=1/\tilde{T}_\text{conv}$.

\begin{center}
\begin{tabular}{crrrr}
\toprule
\toprule
\footnotesize{prescaler} &
\footnotesize{$T_\text{conv}$ ($\mu$s)} & \footnotesize{$\tilde{T}_\text{conv}$ ($\mu$s)} & \footnotesize{$\tilde{f}_\text{conv}$ ($\approx$KHz)} \\
\midrule
2 & 1.8125 & 12.61 & 79.302\\
4 & 3.625 & 16.06  & 62.266 \\
%\rowcolor{grey}
8 & 7.25 & 19.76  & 50.607 \\
16& 14.5 & 20.52  & 48.732 \\
32& 29 & 34.80  & 28.735 \\
64& 58 & 67.89  & 14.729 \\
12& 116 & 114.85 & 8.707  \\
\bottomrule
\end{tabular}
\end{center}

These measurements were made using the \texttt{micros()} function of the
Arduino library API, which has a resolution of about 4~$\mu$s. This might explain
part of the deviation of measured values from the expected values for lower values of prescaler.
8 bit approximation was used, and for obtaining a 10 bit approximation we
can expect an overhead of at least 25\% in conversion time.

It is important to note that the choice for ADC prescaler value limits
the sampling rate of the input signal. As our setup uses a counter's overflow interrupt
to obtain samples from the ADC mechanism, the ADC conversion period must be smaller
than the the PWM's cycle period. Any prescaler choice that leads to a frequency
higher than the PWM's overflow interrupt frequency is valid, but the lower the
prescaler value the lower the quality of the conversion. 


% TODO: comment that input has to be callibrated?


\subsubsection{PWM}
\label{sec:proc-pwm}

From Section \ref{sec:pwm} we can see that in a 16~MHz CPU, an 8 bit counter with prescaler value of $p$ has 
an overflow interrupt frequency of $f_\text{overflow}=10^6/p \times 2^4$~Hz.
% A tabela abaixo é apenas aplicação direta das fórmulas, será que faz sentido?  
Below we can see a table with the overflow interrupt frequency for all possible values of
prescaler:

\begin{center}
\begin{tabular}{crrrr}
\toprule
\toprule
\footnotesize{prescaler} &
\footnotesize{$f_\text{incr}$ (KHz)} &
\footnotesize{$f_\text{overflow}$ (Hz)}  \\
\midrule
1 & 16.000 & 62500 \\
8 & 2.000 & 7812 \\
32 & 500 & 1953 \\
64 & 250 & 976 \\
128 & 125 & 488 \\
256 & 62,5 & 244 \\
1024 & 15,625 & 61 \\
\bottomrule
\end{tabular}
\end{center}

The choice of PWM and ADC prescaler values determine directly the sampling
rate of our DSP system. If we set the ADC prescaler in a way that the ADC
conversion period is smaller than the PWM overflow interrupt period and synchronize
reads from the input with writes to the output, then
the PWM overflow interrupt frequency becomes the DSP system's sample rate. We
will see this with more details in the next section.

For the PWM mechanism, we chose to use Fast PWM mode on
an 8-bit counter with prescaler value of 1. That would give us a sample rate of
62500~Hz, which is enough for representing the audible spectrum. Nevertheless,
if we need more time to compute we may artificially lower the frequency by
only executing the sampling/outputing bit in a fraction of the interrupts. For
our tests, we chose to cut the sample rate in half using the rationale that
the payoff of having more time to compute is larger than the one of ensuring
we can represent the last fifth of the audible spectrum. Therefore, our final choice of sample rate is 31250~Hz, with a sample period
of 32~$\mu$s.

\subsubsection{Putting it all together}

Having chosen a value for the PWM counter size and PWM prescaler, we are left
with the choice for ADC parameters. As noted, it suffices to choose a value
that ensures ADC conversion period is smaller than the desired sample period.
We chose to use 8 bit conversion to match the PWM resolution and to provide for
a faster conversion time. Also, we chose an ADC prescaler value of 8, with a
measured conversion time of 19.76~$\mu$s which, when compared with the
a sample period of 32~$\mu$s ensures that conversion will be finished before
the input ADC is queried for the sample.

Below we can see the code for the \emph{interupt service routine} (ISR) DSP controller function. \texttt{x} is the input buffer, \texttt{ADCH} maps to the ADC register holding the input sample, \texttt{OCR2A} maps to the PWM output regiser and \texttt{y} is the output buffer. Some of the code is index black magic and the rest we comment below.

\begin{lstlisting}
// Timer2 Interrupt Service at 62.5 KHz
ISR(TIMER2_OVF_vect) {
  static boolean div = false;
  div = !div; // divide frequency to 31.25 KHz
  if (div){
    // 1. read from ADC input
    x[ind] = ADCH;
    // 2. write to PWM output
    OCR2A = y[(ind-MIN_DELAY)&(BUFFER_SIZE-1)];
    // 3. signal availability of new sample block
    if ((ind & (BLOCK_SIZE - 1)) == 0) {
      rind = (ind-BLOCK_SIZE) & (BUFFER_SIZE-1);
      dsp_block = true;
    }
    // 4. increment read/write buffer index
    ind++;
    ind &= BUFFER_SIZE - 1;
    // 5. start new ADC conversion
    sbi(ADCSRA,ADSC);
  }
} 
\end{lstlisting}

Note that in step 3 we test if the input index is a multiple of the block size and, if it is, we set a read index \texttt{rind} and signal that there is a new DSP block available for calculation. Meanwhile, the \texttt{loop()} function is running concurrently and will eventually catch that signal and start to work on samples. Finally, we increment buffer indexes and perform the call to start a new ADC conversion by calling \texttt{sbi()}.

\subsection{Benchmarking}

%% Once we have a way to input and output analog audio signals to and from the
%% Arduino board, then we can start to experiment and measure the time it takes
%% to perform common DSP routines in this environment.

We are interested in evaluating the performance of the Arduino board on some common sound processing tasks, in order to gain insight on its real time stream processing capabilities.
Note that our interest lies
in high-level DSP operations; for instance, we'd prefer to know how many simultaneous sinuisoids can
be synthesized in real time rather than how many multiplications
and additions fit between successive DSP blocks (even though the former follows from
the latter).
% TODO: explicar melhor ou sumir com a frase acima. 

Some questions arise immediately from the real time constraint:

\begin{itemize}
    \item What is the maximum amount of DSP operations that can be carried in real
    time?
    \item Which implementation details make a difference?
\end{itemize}

We try to answer these questions by running 3 different DSP algorithms in the
microcontroller environment described in the last section. The chosen tasks are additive synthesis, time-domain convolution and FFT computation, and are discussed in the sequel.

\subsubsection{Additive synthesis}

An additive synthesis is the process of constructing a complex waveform
by adding together several basic waveforms (see Figure \ref{fig:add}).
This technique has been widely used for synthesizing new sounds as well as
resynthesizing signals after they have been processed (e.g. via spectral methods).

\figura{add}{0.20}{Additive synthesis: many basic oscillators governed by
independent amplitude and phase functions are combined to form a complex signal.}

The high level code for a simple additive synthesis can be seen below:

\begin{lstlisting}%[caption=Additive synthesis algorithm.]
for (n = 0; n < N; n++)
{
  angle = 2.0 * M_PI * t;
  y[n] = 0.0;
  for (k = 0; k < numFreqs; k++)
    y[n] += r[k]*sin(f[k] * angle);
  t += 1.0 / SR;
}
\end{lstlisting}

\subsubsection{Time-domain convolution}

Frequency-domain multiplication of spectra correspond to time-domain
convolution of signals, and such an operation allows for some techniques of frequency filtering.
The time-domain implementation of convolution is a widely used technique in
many computer music algorithms, being particularly efficient when the filter order $N$ is small. The general scheme can be seen in Figure
\ref{fig:FIR_Filter}.

\figura{FIR_Filter}{0.45}{Time-domain convolution: the input signal is convolved with the filter's impulse response. This is the general scheme for FIR filtering.}

The high-level code for a time-domain convolution with a FIR filter of order $N$ is:

\begin{lstlisting}
for (k = 0; k < N; k++)
  y[n] += b[k]*x[n-k];
\end{lstlisting}

\subsubsection{Fast Fourier Transform}

The Fast Fourier Transform (FFT) is a clever implementation of the traditional
Fourier Transform that brings its complexity down from $O(n^2)$ to
$O(n\log(n))$, where $n$ is the number of time-domain digital samples or,
equivalently, the number of frequency bins that describe the frequency
spectrum of the signal after the Transform computation. The FFT algorithm
takes advantage of redundancy and symmetry on intermediary steps of the
calculation and is used in many signal processing algorithms. The implementation
code will be given in Section \ref{sec:results-fft} but the general scheme of the FFT can
be seen in Figure \ref{fig:butterfft}.

%-----------------------------------------------------------------------------
%\figura{./img/fft1.pdf}{0.35}{TODO: change caption}


\figura{butterfft}{0.35}{The FFT uses a divide-and-conquer approach and saves
intermediate results to accelerate the calculation of a signal spectrum. The
figure shows one step of an 8-point FFT calculation and how the results map to
frequency bins.}
%-----------------------------------------------------------------------------

\subsubsection{Benchmarking}

Each of the algorithms mentioned in the last sections have different
computational costs in terms of number of integer and floating-point
operations, and quantity/size of memory reads and writes. 

In the context of real time audio processing in Arduino, the algorithms mentioned above
bring natural questions regarding feasibility of processing:

\begin{itemize}
  \item Additive synthesis: what is the maximum number of oscillators that can
  be used to compute a new waveform in real time?
  \item Time-domain convolution: what is the maximum length of a filter that
  can be applied to an audio signal in real time?
  \item FFT: what is the maximum length of an FFT that can be computed in real time?
\end{itemize}

