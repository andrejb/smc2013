%-----------------------------------------------------------------------------
% __  __      _   _               _     
%|  \/  | ___| |_| |__   ___   __| |___ 
%| |\/| |/ _ \ __| '_ \ / _ \ / _` / __|
%| |  | |  __/ |_| | | | (_) | (_| \__ \
%|_|  |_|\___|\__|_| |_|\___/ \__,_|___/
%-----------------------------------------------------------------------------

\section{Methods}
\label{sec:methods}

In order to meet the needs for real time audio processing, the microcontroller
has to be tweaked so we can capture, process and output analog audio. Each of
these tasks can be performed in a variety of ways, and for this examination we
chose to go with the basic functionalities of the platform.

In this investigation, we used an Arduino Duemilanove with an ATmega328P
microcontroller, a very modest version of the platform. It has an 8-bit RISC
central processor, operates with a base frequency of 16~MHz, and has memory
capacity of 32~KB for program storage and 2~KB for random access.

\subsection{Microcontroller's elements}

\setlength{\itemsep}{0em}

To be able to know how to configure the platform to suit our needs, a general
understanding of the inner workings of a microcontroller is needed. The Atmel
Atmel megaAVR series microcontroller is comprised of several components, some
of which are fundamental for our investigation and so will be briefly covered
in this section.

\subsubsection{Registers and interrupts}

The microcontroller's CPU is comprised of an arithmetic logic unit that works
with 32 registers -- portions of memory that provide data for computation as
well as determine the execution flow of the program. An \emph{interrupt} is a
deviation from the execution flow that can be triggered by a variety of events
in the system, usually by setting values of different registers.

In our case, interrupts are of extreme value as they are the low level
structures that allow us to execute code with a somewhat fixed frequency (at
least if we assume that the clock frequency is fixed in relation with real
time).

\subsubsection{Clocks}

Clocks provide the frequencies in which the different parts of the
microcontroller work. They are basically either emitters or dividers of square
wave signals that provide the frequency of operation of the CPU, the ADC,
the memory access, etc. Possible sources of clock frequencies are crystal
and RC oscillators.

A useful concept associated with clocks is the one of a \emph{prescaler}.
Prescalers are dividers for clock frequencies that either actually lower the
frequency of a clock or at least trigger specific interrupts on a power of two
fraction of a clock's frequency.

The \emph{system clock} provides the system's base frequency of operation. In
our study, we make use of the system clock prescaler to control the PWM
frequency that drives our DSP mechanism, as we will see in the section
\ref{sec:pwm}. Another important clock is the \emph{I/O} clock used for
feeding a frequency to most of the input/output mechanisms.

\subsubsection{Input and output ports}

Microcontrollers can receive and emit digital signal through I/O ports, which
in the case of the the Arduino board are conveniently mounted in a way it is
easy to plug other components and boards. These ports are read from and
written to according to frequencies governed by the I/O, ADC or other
indepedent clocks,



\begin{itemize}
    \item CPU: arithmetic unit and registrers (16~MHz - 8 bits).
    \item Interruptions.
    \item Memory: flash (32~KB), SRAM (2~KB), and EEPROM (1~KB).
    \item System clocks (several sources, prescalers).
    \item Power management.
    \item Digital input/output ports.
    \item Counters (with PWM support).
    \item Serial interface.
    \item ADC conversion.
    \item Boot-loader and autoprogramming.
\end{itemize}


\subsection{Audio in: ADC}

Rather than providing the microcontroller with digital data, our setup
includes analog-to-digital conversion of audio input through the
microcontroller's ADC pins.

\begin{itemize}
    \item Sample and hold.
    \item Successive approximation.
    \item resolution (8 or 10 bits)
    \item Conversion time (13 - 250 $\mu$s).
    \item Dedicated clock, noise reduction.
    \item Automatic or manual conversion.
\end{itemize}


Measurement of conversion time, using different values for prescaler:

\begin{center}
\begin{tabular}{crrrr}
\toprule
\toprule
\footnotesize{prescaler} & \footnotesize{$f_\text{ADC}$ (KHz)} &
\footnotesize{$T_\text{ADC}$ ($\mu$s)} & \footnotesize{$\tilde{T}_\text{conv}$ ($\mu$s)} & \footnotesize{$\tilde{f}_\text{conv}$ ($\approx$Hz)} \\
\midrule
2 & 8.000 & 0,125 & 12,61 & 79.302\\
4 & 4.000 & 0,25 & 16,06  & 62.266 \\
%\rowcolor{grey}
8 & 2.000 & 0,50 & 19,76  & 50.607 \\
16& 1.000 & 1 & 20,52  & 48.732 \\
32& 500 & 2 & 34,80  & 28.735 \\
64& 250 & 3 & 67,89  & 14.729 \\
128& 125 & 8 & 114,85 & 8.707  \\
\bottomrule
\end{tabular}
\end{center}
Obs:
\begin{itemize}
  \item The resolution of the \texttt{micros()} function is of 4~$\mu$s.
  \item Conversion perido $\approx$ $14.5 \times T_\text{ADC}$. 
  \item $R=44100$~Hz $\Rightarrow$ $T_\text{amostra} \approx 22.67~\mu$s.
  \item $R=31250$~Hz $\Rightarrow$ $T_\text{amostra} = 32.00~\mu$s.
\end{itemize}

ADC parameter choice:

\begin{itemize}
    \item Left-aligned conversion (8 bits).
    \item Prescaler equals to 8.
\end{itemize}

\subsection{Audio out: PWM}
\label{sec:pwm}

For outputting the result, PWM conversion is used, followed by a simple
lowpass filter.

PWM characteristics:

\begin{itemize}
    \item 6 output channels.
    \item Different modes of operation (fast pwm and phase correct pwm).
    \item prescaler.
    \item Two 8-bit counters and one 16-bit counter.
    \item Overflow interrupt.
\end{itemize}

\figura{pwmmeu}{0.40}{TODO: change captions.}
%\figura{pwm2}{0.35}{TODO: change captions.}
\figura{Pwm}{0.45}{TODO: change captions.}

Explain how we use PWM interrupts to determine the frequency of operation of
our DSP system (and other possibilities of how this could be done).

Frequency of operation given an 8-bit counter and different values for
prescaler:

\begin{center}
\begin{tabular}{crrrr}
\toprule
\toprule
\footnotesize{prescaler} &
\footnotesize{$f_\text{incr}$ (KHz)} &
\footnotesize{$f_\text{overflow}$ (Hz)}  \\
\midrule
1 & 16.000 & 62.500 \\
8 & 2.000 & 7.812 \\
32 & 500 & 1.953 \\
64 & 250 & 976 \\
128 & 125 & 488 \\
256 & 62,5 & 244 \\
1024 & 15,625 & 61 \\
\bottomrule
\end{tabular}
\end{center}

Chose parameters:

\begin{itemize}
  \item \emph{Fast PWM}.
  \item 8-bit counter.
  \item Prescaler value of 1.
  \item Overflow frequency: 16~MHz / 1 / $2^8$ = 62500~Hz.
  \item Sample generation frequency: 31250~Hz.
\end{itemize}

\subsection{Processing}

\begin{lstlisting}
/* 1. read from ADC input */
x[ind] = ADCH;

/* 2. write to PWM output */
OCR2A = y[(ind-MIN_DELAY)&(BUFFER_SIZE-1)];

// 3. signal availability of new sample block
if ((ind & (BLOCK_SIZE - 1)) == 0) {
  rind = (ind-BLOCK_SIZE) & (BUFFER_SIZE-1);
  dsp_block = true;
}

/* 4. increment read/write buffer index */
ind++;
ind &= BUFFER_SIZE - 1;

/* 5. start new ADC conversion */
sbi(ADCSRA,ADSC); 
\end{lstlisting}



\subsection{Benchmarking}

Once we have a way to input and output analog audio signals to and from the
Arduino board, then we can start to experiment and measure the time it takes
to perform common DSP routines in this environment.

For real time purposes, there's a restriction for the ammount it takes to
calculate samples' values and have them ready for playback. If the DSP block
period is of $N$ samples and the sample rate is $R$~Hz, then the DSP cycle
period is given by $T_{DSP}=\frac{N}{R}$ seconds.

Some questions arise immediately from such a constraint:

\begin{itemize}
    \item What is the maximum amount of operations that can be carried in real
    time?
    \item Which implementation details make a difference?
    \item What is the quality of the resulting audio signal?
\end{itemize}

\subsubsection{Additive synthesis}

An additive synthesis is the process of constructing a more complex waveform
by adding together several more basic waveforms (see Figure \ref{fig:add}).
This technique has been widely used for synthesizing new sounds as well as
resynthesizing signals after they went through some computational procedure.

\figura{add}{0.20}{Additive synthesis.}


The high level code for a simple additive synthesis can be seen below:

\begin{lstlisting}[caption=Additive synthesis algorithm.]
for (n = 0; n < N; n++)
{
  angle = 2.0 * M_PI * t;
  y[n] = 0.0;
  for (k = 0; k < numFreqs; k++)
    y[n] += r[k]*sin(f[k] * angle);
  t += 1.0 / SR;
}
\end{lstlisting}

\subsubsection{Time-domain convolution}

Frequency-domain multiplication of spectra correspond to time-domain
convolution of signals, and such an operation allows for frequency filtering.
The time-domain implementation of convolution is a widely used technique in
many computer music algorithms and the general scheme can be seen in Figure
\ref{fig:FIR_Filter}.

\figura{FIR_Filter}{0.35}{TODO: change caption}

The high-level code for a time-domain convolution is:

\begin{lstlisting}
for (k = 0; k < N; k++)
  y[n] += b[k]*x[n-k];
\end{lstlisting}

\subsubsection{Fast Fourier Transform}

The Fast Fourier Transform (FFT) is a clever implementation of the traditional
Fourier Transform that brings its complexity down from $O(n^2)$ to
$O(n\log(n))$, where $n$ is the number of time-domain digital samples or,
equivalently, the number of frequency bins that describe the frequency
spectrum of the signal after the Transform computation. The FFT algorithm
takes advantage of redundancy and symetry on intermediary steps of the
calculation and is used in many signal processing algorithms (see Figure
\ref{fig:butterfft}).

%-----------------------------------------------------------------------------
%\figura{./img/fft1.pdf}{0.35}{TODO: change caption}


\figura{butterfft}{0.35}{The FFT uses a divide-and-conquer approach and saves
intermediate results to accelerate the calculation of a signal spectrum. The
figure shows one step of an 8-point FFT calculation and how the results map to
frequency bins.}
%-----------------------------------------------------------------------------

\subsubsection{Benchmarking}

Each of the algorithms mentioned in the last sections have different
computational costs in terms of number of integer and floating-point
operations, and quantity of memory reads and writes (and amount of memory
accessed in each of these operations). 

In the context of real time audio processing in the Arduino, the algorithms mentioned above
bring natural questions regarding feasibility of processing:

\begin{itemize}
  \item Additive synthesis: what is the maximum number of oscillators that can
  be used to compute a new waveform in real time?
  \item Time-domain convolution: what is the maximum length of a filter that
  can be applied to an audio signal in real time?
  \item FFT: what is the maximum length of an FFT that can be computed in real time?
\end{itemize}

