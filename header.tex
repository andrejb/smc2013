%-----------------------------------------------------------------------------
% general document control
%\documentclass[11pt,a4paper,notitlepage]{article}
%\documentclass[journal]{ieeetr}

% journal specifics
\documentclass{article}
\usepackage{smc2012} 
\usepackage{times}
\usepackage{ifpdf}
\usepackage[english]{babel}
\usepackage{cite}
%\usepackage{mathptmx} 

% section counter
%\setcounter{secnumdepth}{1} 

% localization
\usepackage[utf8x]{inputenc}
%\usepackage[brazil]{babel}
%\usepackage{abntex}


% glossary
%\usepackage{glossaries}
%\makeglossaries
%\input{glossario}

% bibliography
%\usepackage[colorlinks,citecolor=blue,urlcolor=brown]{hyperref}
%\usepackage[alf]{abntcite}
%\usepackage[square,comma]{natbib}
%\usepackage[fixlanguage]{babelbib}
%\selectbiblanguage{portugues}


% embellishment
\usepackage{amssymb,amsmath,amsfonts}
\usepackage{booktabs}
%\numberwithin{equation}{section}
%\usepackage{enumerate}
%\usepackage[top=2.5cm, bottom=2.5cm, left=2.5cm, right=2.5cm]{geometry}
%\usepackage{xspace}
%\usepackage{colortbl}
%\usepackage{caption}
%\usepackage{mathdots}
%\usepackage{fancyhdr}
\usepackage{enumitem}
\setlist{nolistsep}


% figures
%\usepackage{graphicx}
%\usepackage{epsfig}

% math
\usepackage{amsthm}


%\usepackage{epstopdf}
%\usepackage[figure,table]{hypcap}	% corrects the hyper-anchor of figures/tables

% todo
%\usepackage[bordercolor=white,backgroundcolor=yellow!30,linecolor=black,colorinlistoftodos]{todonotes}
%\newcommand{\lembrete}[1]{\todo[color=cyan!20,inline]{\ensuremath{\rhd} #1}}


% new commands
%\newcommand{\FFT}{\texttt{FFT}\xspace}
%\newtheorem{definicao}{Definição}
%\newtheorem{exemplo}{Exemplo}
%\newtheorem{problema}{Problema}
%\newtheorem{algoritmo}{Algoritmo}
%\newtheorem{sistema}{Sistema}
\newcommand{\jclass}[1]{\texttt{\textmd{#1}}}

% hiphenation
\hyphenation{a-mount}

\usepackage{listings}
\usepackage{color}

\definecolor{dkgreen}{rgb}{0,0.4,0}
\definecolor{gray}{rgb}{0.5,0.5,0.5}
\definecolor{mauve}{rgb}{0.58,0,0.82}
\definecolor{LightCyan}{rgb}{0.7,0.9,0.9}
 
\lstset{ %
  language=C,                % the language of the code
  basicstyle=\ttfamily\footnotesize,           % the size of the fonts that are used for the code
  %numbers=left,                   % where to put the line-numbers
  numberstyle=\tiny,%\color{gray},  % the style that is used for the line-numbers
  stepnumber=1,                   % the step between two line-numbers. If it's 1, each line 
                                  % will be numbered
  numbersep=5pt,                  % how far the line-numbers are from the code
  %backgroundcolor=\color{white},      % choose the background color. You must add \usepackage{color}
  showspaces=false,               % show spaces adding particular underscores
  showstringspaces=false,         % underline spaces within strings
  showtabs=false,                 % show tabs within strings adding particular underscores
  frame=single,                   % adds a frame around the code
  %rulecolor=\color{black},        % if not set, the frame-color may be changed on line-breaks within not-black text (e.g. commens (green here))
  tabsize=2,                      % sets default tabsize to 2 spaces
  captionpos=b,                   % sets the caption-position to bottom
  breaklines=false,                % sets automatic line breaking
  breakatwhitespace=false,        % sets if automatic breaks should only happen at whitespace
  title=\lstname,                   % show the filename of files included with \lstinputlisting;
                                  % also try caption instead of title
  %keywordstyle=\color{blue},          % keyword style
  %commentstyle=\color{dkgreen},       % comment style
  %stringstyle=\color{mauve},         % string literal style
  escapeinside={\%*}{*)},            % if you want to add a comment within your code
  morekeywords={*,...},               % if you want to add more keywords to the set
  %xleftmargin=1em,
}


\newcommand{\figura}[3]{
\begin{figure}[h]
\begin{center}
\includegraphics[width=#2\textwidth]{#1}
\caption{#3}
\label{fig:#1}
\end{center}
\end{figure}
}
