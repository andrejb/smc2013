%-----------------------------------------------------------------------------
%  ___       _                 _            _   _             
% |_ _|_ __ | |_ _ __ ___   __| |_   _  ___| |_(_) ___  _ __  
%  | || '_ \| __| '__/ _ \ / _` | | | |/ __| __| |/ _ \| '_ \ 
%  | || | | | |_| | | (_) | (_| | |_| | (__| |_| | (_) | | | |
% |___|_| |_|\__|_|  \___/ \__,_|\__,_|\___|\__|_|\___/|_| |_|
%-----------------------------------------------------------------------------

\section{Introduction}


Arduino is the name of a hardware and software project started in 2005 which
aims to simplify the interface of electric-electronic devices with a
microcontroller. It evolved from the Processing software IDE (2001) and the
Wiring software and hardware prototyping platform (2003). Hardware, software
and documentation designs are published under free licenses (Creative Commons
BY-SA 2.5, GPL/LGPL and CC BY-SA 3.0 respectively) and a large community has
grown to provide code and support for newcomers. Nowadays, many Arduino
hardware designs are available and range from more limited 8-bit
microcontrollers to fully featured 32-bit ARM CPUs. Besides, other
advantages of Arduino for academic and artistic use are its mobility (because
of its low power needs and possibility of running on batteries for hours, if
not days depending on the use), expandability (because of its standardized
interface for attaching so called hardware \emph{shields}) and price (selling
for under 20 US dollars online).

Despite all these advantages, the platform has a somewhat limited processing
power when compared to standard processors available in the market for
specific or general uses. In this work, we aim to systematically expose the
platform's possibilities for carrying real time digital audio processing tasks
so there can be more accurate elements to be taken into account when making
the choice for a platform.


\subsection{Related work}

Arduino has been experimentally used for real time audio processing for sampling audio and control signals with an effective rate of 15.250~KHz \cite{arduinodsp}, and provided the base for our investigation. Also, an ALSA audio driver was implemented to use the Arduino Duemillanove \cite{Dimitrov:2011} as a full-duplex, mono, 8-bit 44.1~KHz sound card under linux.

%\subsection{Text organization}

