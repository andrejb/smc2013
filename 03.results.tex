%-----------------------------------------------------------------------------
% ____                 _ _       
%|  _ \ ___  ___ _   _| | |_ ___ 
%| |_) / _ \/ __| | | | | __/ __|
%|  _ <  __/\__ \ |_| | | |_\__ \
%|_| \_\___||___/\__,_|_|\__|___/
%-----------------------------------------------------------------------------

\section{Results}
\label{sec:results}

%-----------------------------------------------------------------------------
%     _       _     _ _ _   _             ____              _   _     
%    / \   __| | __| (_) |_(_)_   _____  / ___| _   _ _ __ | |_| |__  
%   / _ \ / _` |/ _` | | __| \ \ / / _ \ \___ \| | | | '_ \| __| '_ \ 
%  / ___ \ (_| | (_| | | |_| |\ V /  __/  ___) | |_| | | | | |_| | | |
% /_/   \_\__,_|\__,_|_|\__|_| \_/ \___| |____/ \__, |_| |_|\__|_| |_|
%                                               |___/
%-----------------------------------------------------------------------------

\subsection{Additive synthesis}

The first experiment tries to answer the question of how many oscillators can
be used when performing real time additive sinthesis inside the platform. In
the beginning of the DSP cycle, an additive synthesis algorithm is run using a
determined number of oscillators and the mean of the synth time is taken over
ten million measurements. Block sizes used had 32, 64 and 128 samples (more
showed to be infeasible in real time) and the number of oscillators was
increased until the DSP cycle period was exceeded.

The first result has to do with the use of loop structures. Because looping
usually requires incrementing and testing a variable in each iteration, the
use of one loop structure may have strong influence in the amount of
oscillators that can be used in real time for additive synthesis inside the
Arduino.  Figures \ref{fig:sinesum-comparison-for} and
\ref{fig:sinesum-comparison} show the amount of oscilators feasible by,
respectivelly, making use of a loop and making use of inline code. By removing
loops we were able to increase from 8 oscillators to at least 13, or 14 in the
case of block size of 128 samples.


\figura{sinesum-comparison-for}{0.45}{Results for blocks of different
sizes.}

\figura{sinesum-comparison}{0.45}{Results for blocks of different
sizes.}

In any DSP algorithm that works over a block of samples there may be at least
one loop structure: the one that loops over all samples of the block. It could
be eliminated but the cost of doing so would implicate having to recompile the
code to change the length of the block, which is highly inconvenient. We apply
the same rationale to the algorithms implemented in the experiments, and
therefore we make use of loop structures, although the least we can. That has
the good side-effect of allowing us to run experiments by variating parameters
values dinamically.

While implementing this experiment, a first attempt was made using the
standard API \texttt{sin()} function. As that showed to be infeasible in real
time, we focused on table lookup implementations. At this point we noticed
that even the smallest implementation difference can have large impact on the
results. Therefore, we decided to test and plot the results for different
numbers and types of operations.

Two parameters are used to calculate the value of each oscillator: phase and
amplitude. Phase is handled by updating the index for sine table reads, and
then the amplitude has to be multiplied by the value obtained by the lookup.
Floating point operations are also extremelly expensive in the platform we are using, so 
we implemented 3 different ways of multiplying the amplitude: (1) by using one integer multiplication and one integer division (2 integer operations), (2) by using only one integer division (1 integer operation), and (3) by using variable bit pading for performing bitwise power of 2 division or multiplication.
% TODO: epxlicar melhor acima.
Figure \ref{fig:operations-128-31250} shows the time taken by the calculation
of the additive synthesis algorithm for different number of oscillators using
the different types of operations. By making use of lower level operations
(that achieve less precise results) and inline code we were able to raise the number of
oscillators from 3 (when using 2 integer operations and a for loop) to 15 (when using a variable
pad and inline code).


\figura{operations-128-31250}{0.45}{Time taken for additive synthesis
algorithm with block size of 128 samples, using different number and kinds of operations and variable number of oscillators.}

%\figura{./img/frequencies-128-1.pdf}{0.35}{Results for different sample rates.}

%Result summary:
%
%\begin{center}
%\begin{tabular}{rcccc}
%\toprule
%\toprule
%block size  & 2op & 1op & pad+for & pad \\
%\midrule
%32  & 1 & 3 & 7 & 13 \\
%64  & 1 & 3 & 7 & 13 \\
%128 & 1 & 3 & 7 & 14 \\
%\bottomrule
%\end{tabular}
%\end{center}


%-----------------------------------------------------------------------------
%   ____                      _       _   _             
%  / ___|___  _ ____   _____ | |_   _| |_(_) ___  _ __  
% | |   / _ \| '_ \ \ / / _ \| | | | | __| |/ _ \| '_ \ 
% | |__| (_) | | | \ V / (_) | | |_| | |_| | (_) | | | |
%  \____\___/|_| |_|\_/ \___/|_|\__,_|\__|_|\___/|_| |_|
%                                                       
%-----------------------------------------------------------------------------
\subsection{Time-domain convolution}

Our second experiment tries to clarify what is the maximum size of a FIR
filter that can be applied in real time to an input signal by use of
time-domain convolution algorithms. Following lessons learned on the first
experiment, we implemented the filtering loop using different types of operations for multiplying each coefficient by the sample values: (1) using one integer division and one integer division, (2) using variable pad, and (3) using a constant hardcoded pad. The results for each of these implementations can be seen in Figures \ref{fig:convolution-comparison-div}, \ref{fig:convolution-comparison-vpad} and \ref{fig:convolution-comparison-cpad} respectivelly. This experiment was run with a sample rate of 31250~Hz and block sizes of 32, 64, 128 and 256 samples.

% Temporariamente desabilitando para eu poder compilar... não tenho essa figura
%\figura{convolution-comparison-div}{0.45}{Time-domain convolution using 2 integer operations. In this case, the maximum order of the filter is 1 for all block sizes.}

\figura{convolution-comparison-vpad}{0.45}{Time-domain convolution using variable pad. In this case, the maximum order of the filter is 7 for all block sizes.}

\figura{convolution-comparison-cpad}{0.45}{Time-domain convolution using constant pad. In this case, the maximum order of the filter is 13 for block sizes of 32 and 64, and 14 for block sizes of 128 and 256.}

Results again show that small implementation differences make a big difference on computing power. When using integer division, the maximum order obtained for the filter was 1, while by using a variable pad the order raised to 7 and with constant pad we could read an order equal to 13 or 14 in some cases. 


%Ordem máxima do filtro FIR em cada cenário ($R=31.250$~Hz):
%
%\begin{center}
%\begin{tabular}{rccc}
%\toprule
%\toprule
%\footnotesize{block size}  & \footnotesize{multiplicação} & \footnotesize{pad variável} & \footnotesize{pad constante} \\
%\midrule
%32  & 1 & 7 & 13 \\
%64  & 1 & 7 & 13 \\
%128 & 1 & 7 & 14 \\
%256 & 1 & 7 & 14 \\
%\bottomrule
%\end{tabular}
%\end{center}

%-----------------------------------------------------------------------------
%{Exemplo: moving average}
%\figura{./img/moving.pdf}{0.35}{TODO: change caption}


%-----------------------------------------------------------------------------
%  _____ _____ _____ 
% |  ___|  ___|_   _|
% | |_  | |_    | |  
% |  _| |  _|   | |  
% |_|   |_|     |_|  
%
%-----------------------------------------------------------------------------

\subsubsection{FFT}
\label{sec:results-fft}

The third experiment is concerned with the maximum length of an FFT that can
be computed in real time inside an Arduino. In this case we chose to evaluate a standard implementation of the FFT without further modifications. The macro defining the call for one-dimensional fast fourier algorithm is:

\begin{lstlisting}
#define four1(data, nn, isign)
  int i, j, n, mmax, m, istep;
  float wtemp, wr, wpr, wpi, wi, theta, tempr,
        tempi;
  n = nn << 1;
  j = 1;
  for(i=1; i<n; i+=2) {
    if(j > i) {
      float temp;
      temp = data[j];
      data[j] = data[i];
      data[i] = temp;
      temp = data[j+1];
      data[j+1] = data[i+1];
      data[i+1] = temp;
    }
    m = n >> 1;
    while(m >= 2 && j > m) {
      j -= m;
      m >>= 1;
    }
    j += m;
  }
  mmax = 2;
  while(n > mmax) {
    istep = (mmax << 1);
    theta = isign*(6.28318530717959/mmax);
    wtemp = sin(0.5*theta);
    wpr = -2.0*wtemp*wtemp;
    wpi = sin(theta);
    wr = 1.0;
    wi = 0.0;
    for(m=1; m<mmax; m+=2) {
      for(i=m;i<=n;i+=istep) {
        j = i+mmax;
        tempr = wr*data[j]-wi*data[j+1];
        tempi = wr*data[j+1]+wi*data[j];
        data[j] = data[i] - tempr;
        data[j+1] = data[i+1] - tempi;
        data[i] += tempr;
        data[i+1] += tempi;
      }
    wr = (wtemp=wr)*wpr-wi*wpi+wr;
    wi = wi*wpr+wtemp*wpi+wi;
    }
    mmax = istep;
  }
\end{lstlisting}

It turned out that calculating an FFT using the same sample rate we used in the other experiments (31250~Hz) was infeasible, so we had to tweak the microcontroller's parameters to reach a state where we had a longer DSP cycle period for the same amount of samples and the FFT was indeed feasible. By measuring the amount of time taken to compute the FFT given the number of samples, we could determine that the maximum FFT frequency for a 256 samples block is of about 2335~Hz. So by raising the PWM prescaler value to 32, we could reach a sample rate of about 1953~Hz. 

Figure \ref{fig:fft2} shows the FFT analysis time at a sample rate of 1953~Hz for different block sizes. We can see that in this scenario the maximum block size for which an FFT can be computed in real time in our DSP setup in the Arduino is 256 samples. This was expected because we actually forced a sample rate small enough so the 256 samples FFT was feasible. Note that, even though we can actually perform the FFT for block sizes smaller or equal to 256, there's not much time left for doing anything else with these results. An additive synthesis for reconstructing the signal, for example, is unfeasible as the maximum number of oscillators we could use was 14 (by restricting the type and number of operations), while here we would need the same number of oscillators as the number of samples in the block size.

\figura{fft2}{0.45}{TODO: change caption}
%\figura{fft}{0.35}{TODO: change caption}




